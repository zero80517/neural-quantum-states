% Используются листы формата А4, односторонняя печать, чистовая версия с кеглем 14 пунктов для отчета 
\documentclass[a4paper, oneside, final, 14pt]{extreport}

% дальше идет непонятная дичь, которая делает крутое оглавление со всеми точками и циферками
\usepackage{titlesec}
\usepackage{tocbasic}

\titleformat{\section}
{\large\bfseries}{\thesection}{1em}{}
\titleformat{\subsection}
{\large\bfseries}{\thesubsection}{1em}{}

\DeclareTOCStyleEntry[
    linefill=\bfseries\TOCLineLeaderFill,beforeskip=2pt,entrynumberformat=\chapterprefixintoc,dynnumwidth
]{tocline}{chapter}
\newcommand\chapterprefixintoc[1]{#1~}% <- added

\titleformat{\chapter}[hang]{\centering\huge\bfseries}{\thechapter}{1em}{}
\titlespacing*{\chapter}{0pt}{-3em}{2em}
\titlelabel{\thetitle\quad}
\renewcommand{\thechapter}{Глава \arabic{chapter}.}
\renewcommand{\thesection}{\arabic{chapter}.\arabic{section}}
\renewcommand{\theequation}{\arabic{chapter}.\arabic{equation}}
\renewcommand{\thetable}{\arabic{chapter}.\arabic{table}}
\renewcommand{\thefigure}{\arabic{chapter}.\arabic{figure}}

% Подрубить плавающие иллюстрации
\usepackage{graphicx}

% Все рисунки лежат в папке pictures, so можно тырить без указания папки
\graphicspath{{pictures/}}
\DeclareGraphicsExtensions{.pdf,.png,.jpg}  % тривиально

% Заменить ": " на ". " в рисунках согласно российским типографским правилам. 
% Центрировать подпись к таблицам и рисункам. Название таблиц расположить сверху
% \usepackage[format=plain, labelsep=period, justification=centering, tableposition=top]{caption}  
\usepackage[format=plain, labelsep=period, tableposition=top]{caption}

% Подрубить крутые пакеты для не менее крутых математических формул
\usepackage{amssymb,amsfonts,amsmath,amsthm}

% Настройка отступов
\usepackage[left=3cm,right=1.5cm,top=2cm,bottom=2cm]{geometry}

% Убрать пробел после запятой в формулах
\usepackage{icomma}

% Подрубить мультиязыковой пакет + настроить формат шрифта (называется последнее иначе)
\usepackage{polyglossia}
\setmainlanguage[spelling=modern]{russian}  % spelling=modern все еще под вопросом
\setotherlanguage{english}
\setmonofont{Courier New}
\newfontfamily\cyrillicfonttt{Courier New}         
\defaultfontfeatures{Ligatures=TeX}         % Ligatures=TeX все еще под вопросом
\setmainfont{Times New Roman}
\newfontfamily\cyrillicfont{Times New Roman}
\setsansfont{Arial}
\newfontfamily\cyrillicfontsf{Arial}

% Возможно настраивать english избыточно, так как такие символы, вроде \hbar теряют смысл.
% Что с остальными символами я правда не знаю. Так что просто для напоминания (20.06.2020)
% \usepackage{polyglossia}
% \setmainlanguage[spelling=modern]{russian}  % spelling=modern все еще под вопросом
% \setotherlanguage{english}
% %\setmonofont{Courier New}
% \newfontfamily\cyrillicfonttt{Courier New}         
% \defaultfontfeatures{Ligatures=TeX}         % Ligatures=TeX все еще под вопросом
% %\setmainfont{Times New Roman}
% \newfontfamily\cyrillicfont{Times New Roman}
% %\setsansfont{Arial}
% \newfontfamily\cyrillicfontsf{Arial}

% Отделять первую строку раздела абзацным отступом, согласно ru правилам
\usepackage{indentfirst}

% ru правила требуют только французские пробелы
\frenchspacing

% Настроить оформление списка литературы
\usepackage[square,numbers]{natbib}

% Впечатать упоминание списка литературы в содержании
\usepackage[nottoc]{tocbibind}

% Оформить алгоритмы и руссифицировать (прости господи) их
\usepackage[ruled, linesnumbered, noline]{algorithm2e}
\SetNlSty{textbf}{}{:}
\DontPrintSemicolon

\renewcommand{\thealgocf}{}
\renewcommand{\listalgorithmcfname}{Список алгоритмов}
\renewcommand{\algorithmcfname}{Алгоритм}
\renewcommand{\algorithmautorefname}{алгоритм}
\renewcommand{\algorithmcflinename}{линия}
%\renewcommand{\algocf@typo}{}%
%\renewcommand{\@algocf@procname}{Procedure}%
%\renewcommand{\@algocf@funcname}{Function}%
\renewcommand{\procedureautorefname}{procedure}%
\renewcommand{\functionautorefname}{function}%
%\renewcommand{\algocf@languagechoosen}{english}%

\SetKwHangingKw{KwHData}{Data$\rightarrow$}
\SetKwInput{KwIn}{Input}%
\SetKwInput{KwOut}{Output}%
\SetKwInput{KwData}{Data}%
\SetKwInput{KwResult}{Result}%
\SetKw{KwTo}{to}
\SetKw{KwRet}{return}%
\SetKw{Return}{return}%
\SetKwBlock{Begin}{begin}{end}%
\SetKwRepeat{Repeat}{повторять}{до тех пор, пока}

\SetKwIF{If}{ElseIf}{Else}{if}{then}{else if}{else}{end if}%
\SetKwSwitch{Switch}{Case}{Other}{switch}{do}{case}{otherwise}{end case}{end switch}%
\SetKwFor{For}{for}{do}{end for}%
\SetKwFor{ForPar}{for}{do in parallel}{end forpar}
\SetKwFor{ForEach}{foreach}{do}{end foreach}%
\SetKwFor{ForAll}{forall}{do}{end forall}%
\SetKwFor{While}{while}{do}{end while}%

% Добавить крутую графику
\usepackage{tikz}  
\usetikzlibrary{graphs, graphs.standard,matrix}

% Использовать для создания гиперссылок, если понадобиться. Добавлять только в конце!
\usepackage{hyperref}

% не менять стиль написания ссылок
\urlstyle{same}