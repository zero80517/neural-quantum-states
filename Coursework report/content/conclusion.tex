\chapter*{Заключение}
\addcontentsline{toc}{chapter}{Заключение}

В ходе данной работы был рассмотрен подход представления квантовых состояний системы спинов как состояний ограниченной машины Больцмана. 
В качестве проверки точности данного подхода была рассмотрена задача по нахождению вектора основного состояния для классических гамильтонианов модели Изинга поперечного поля и модели антиферромагнетика Гейзенберга, а также изотропной части гамильтониана магнитного кластера $\text{V}_{15}$, посредством вариационного метода Монте-Карло.
В ходе работы в качестве методов решения поставленной задачи были использованы различные модификации метода градиентного спуска.
В результате обучения ограниченной машины Больцмана получилось достичь относительной погрешности $<1\%$, что указывает на наличие потенциала в данной области исследований.

Основной проблемой при решении поставленной задачи явилась свойственная рассмотренным оптимизационным алгоритмам проблема подбора оптимальной скорости обучения для достижения точных результатов за относительно небольшое число шагов обучения.
Данная проблема является краеугольной в машинном обучении, поэтому мы рассмотрели несколько алгоритмов ее решения.
Именно за адаптивными методами обучения лежит дальнейшее развитие использования искусственных нейронных сетей в области физических исследований, так как для разных физических условий подбор параметров обучения будет затратным по времени вычислений занятием.
Поэтому использование алгоритмов, самостоятельно выбирающих скорость обучения, будет уместным для различных физических задач.
Естественно, для более точного результата требуется увеличить число параметров ограниченной машины Больцмана.
Но тогда нужны и более надежные методы регуляризации для борьбы с переобучением искусственной нейронной сети.

Все эти результаты показывают, что необходимо в первую очередь выявить наиболее важные критерии сходимости искусственной нейронной сети в ходе обучения, дабы за несколько шагов определить диапазон параметров обучения для сходимости метода.
Таким образом, необходимо использование дополнительной искусственной нейронной сети, для обучения на нахождение диапазона параметров обучения, приводящих к устойчивому результату.
При этом основным критерием является быстрая обучаемость на небольшой выборке.