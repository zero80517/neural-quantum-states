\chapter*{Введение}
\addcontentsline{toc}{chapter}{Введение}

С увеличением объема данных и вычислительных мощностей методы машинного обучения все больше используются как средство для анализа. 
Терабайты разнообразной информации хранят незримые для человека закономерности, которые желательно выявлять для более глубокого понимания процессов, происходящих в окружающем нас мире. 
Методы машинного обучения являются, по-сути, единственным инструментом познания окружающего мира посредством анализа больших данных.
Огромное количество данных непосредственно производится в научной сфере, вследствие чего применение методов машинного обучения для решения физических задач заинтересовало множество научных групп, связанных с информационными технологиями.

Так, недавно были проведены исследования по обнаружению нетривиальных состояний материи посредством машинного обучения \cite{carrasquilla2017machine}.
Эти и подобные исследования позволяют предположить наличие потенциала в приложении машинного обучения к физическим исследованиям и его дальнейшее использование в будущем \cite{butler2018machine}.

Особый интерес представляет применение искусственных нейронных сетей для приближенного решения стационарного уравнения Шредингера многочастичных систем. В данной работе будет рассматриваться решение последнего для основного состояния системы спинов $S=1/2$:

\begin{equation*}
\hat{\mathcal{H}} | \psi_0 \rangle=E_0 | \psi_0 \rangle
\end{equation*}

Для системы спинов размерность гильбертова пространства растет как $2^{N}$, где $N$~--- число спинов. 
Так как для вычисления средних значений наблюдаемых необходимо считать матричные элементы на состояниях всего гильбертова пространства, то вычисление оных может быть нереализуемо в силу конечной вычислительной мощности и вместимости памяти. 
Так, для $N=40$ спинов необходимо порядка 10 терабайтов памяти для полного задания состояния спиновой системы.
Для моделирования состояния подобного ансамбля спинов необходимо характеризовать данную систему меньшим, но достаточным числом параметров.

Наиболее известны два метода решения таких задач: стохастический метод (например, квантовый метод Монте-Карло \cite{ceperley1986quantum}) и метод эффективного сжатия гильбертова пространства системы спинов (например, ренормгруппа матрицы плотности \cite{white1992density}).
В недавней статье \cite{carleo2017solving} был рассмотрен новый метод решения данной задачи, являющийся эффективным объединением двух предыдущих методов. 
Суть данного метода состоит в аппроксимации вектора основного состояния системы спинов \textit{искусственной нейронной сетью}, которая представляет собой многомерную функцию, являющуюся суперпозицией двух одномерных функций с некоторым количеством неопределенных параметров.
Математические основы достижения данным методом удовлетворительной аппроксимации опираются на известную математическую теорему Колмогорова-Арнольда о представлении произвольной многомерной функции суперпозицией одномерных функций \cite{kolmogorov1956representation}.
Следуя этой теореме, можно подобрать такую непрерывную функцию, что с ее помощью можно будет с произвольной точностью выразить вектор основного состояния. 
Поэтому, подобрав достаточно хорошую функцию, коей мы выбираем согласно \cite{carleo2017solving} \textit{ограниченную машину Больцмана}, можно будет с ее помощью решить данную задачу. 
Этот метод и будет рассмотрен в данной работе.

В качестве критерия точности аппроксимации ограниченной машиной Больцмана вектора основного состояния будет выступать энергия основного состояния системы спинов:

\begin{equation*}
E[\psi]=\frac{\langle \psi | \hat{\mathcal{H}} | \psi \rangle}{\langle \psi | \psi \rangle}
\end{equation*}

Если известно точное значение энергии основного состояния, то его можно сравнить с полученным по ходу \textit{обучения} искусственной нейронной сети.
Обучение искуственной нейронной сети представляет собой процесс подбора таких параметров, которые бы позволяли ей решать поставленную задачу.
Если же точное значение энергии не известно, то точность аппроксимации можно проверить, проверяя устойчивость вычисляемого значения энергии  при использовании различных методов обучения ограниченной машины Больцмана.